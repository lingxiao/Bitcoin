\title{December 29th}

\subsection{Plan to accomplish today}

\begin{itemize}
	\item Build basic Etherium contract following tutorial from source 1. Have a basic prototype running on node.js
	\item consider how to port it to a python version
\end{itemize}

\subsection{Today I learned: how to get started with Ethereum}


\begin{itemize}
	\item install Geth: so I can run ethereum client on computer
		\begin{itemize}	
			\item source: http://candidtim.github.io/ethereum/2016/03/24/ethereum-quick-start.html
		\end{itemize}
	% 
	\item doing this \[geth --testnet --datadir ~/.ethereum-testnet\] will download entire testnet blockchain onto computer
	% 
	\item doing this: \[geth --fast --cache 1024\] downloads a compressed (?) version of blockchain
	% 
	\item Test RPC is a client for testing:
	% 
			testrpc is a Node.js based Ethereum client for testing and development. It uses ethereumjs to simulate full client behavior and make developing Ethereum applications much faster. It also includes all popular RPC functions and features (like events) and can be run deterministically to make development a breeze.
	% 
			While Geth is a full client in GO Language that you can use to connect to the real chain, or start your own testnet server.
	% 
			geth is the the command line interface for running a full ethereum node implemented in Go. It is the main deliverable of the Frontier Release
	% 
	\item mine blocks and earn ether on testnet
	% 
\end{itemize}


\subsection{Sources to Consider}

\begin{itemize}
	\item The ethereum hello world: https://medium.com/@mvmurthy/ethereum-for-web-developers-890be23d1d0c
	\item The ethereum book: https://ethereum.gitbooks.io/frontier-guide/content/ethereum.html
	\item Trending Cryptocurrencies: https://docs.google.com/spreadsheets/d/1r7YFBWGshhXJwXzSUKT61SngmtR8OfvIA08kXe6WJw4/edit#gid=1048393921
	\item 24 hour volume ranking for cryptocurrencies: https://coinmarketcap.com/exchanges/volume/24-hour/
\end{itemize}	


\title{December 30th}

\subsection{Plan to accomplish today}

\begin{itemize}
	\item Build basic Etherium contract following tutorial from source 1. Have a basic prototype running on node.js
	\item consider how to port it to a python version
\end{itemize}

\subsection{Today I learned: how to develop basic Ethereum contract with Geth}

Note, lightweight etherium client development is incomplete, so mobile support is weak.

\begin{itemize}
	\item install geth: \[bash <(curl -L https://install-geth.ethereum.org)\], it's a go based ethereum virtual machine (EVM).
	\item start geth \[geth --dev console\]
	\item list accounts \[eth.accounts\]
	\item create a test account \[personal.newAccount()\]
	\item check balance: \[eth.getBalance(eth.accounts[i])\]
	\item write a contract in Solidity, a Javascript-derived language. It can be compile to bytecode using \[sol-c\] compiler
	\item sol-c compiler is installed globally with \[npm install -g solc\]
	\item Three pieces of software to keep in mind when developing:
 }}		\begin{itemize}
			\item Ethereum Node (Geth server): Runs the node and maintain connection to rest of server, run by running \[geth console\]
			\item Console: allow javascript command-line interface to the server. 
			can open it with \[geth console\], or run \[geth\], then in a separate window do \[geth attach\] Note if we have \[hello.js\], then we can load it into geth console with \[loadScript("path/to/hello.js")\]
			\item Ethereum Wallet (aka Mist): GUI for interfacing with wallets and deploying contracts. Will look for /Users/emunsing/Library/Ethereum/geth.ipc and if it does not find geth.ipc the Wallet will start up a server for the node.
			\item JSON RPC API: lightweight format, it is what is communicated over the network. \[web3.js\] is a library that implents this API.
		\end{itemize}
 	%
 	% 
 	% 
 	\item some interesting sources:
 		\begin{itemize}
 			\item could you get sued? https://docs.google.com/spreadsheets/d/1QxOV2dgxO3C_TyVE0-41ZwLlzPmB-EE1NNshJGuedCU/edit#gid=0
 			\item framework for regulation: https://coincenter.org/entry/framework-for-securities-regulation-of-cryptocurrencies
 			\item http://www.ethdocs.org/en/latest/ethereum-clients/choosing-a-client.html
			\item https://ethereum.stackexchange.com/questions/566/how-to-write-my-first-solidity-hello-world-smart-contract
			\item https://ethereum.org/greeter
			\item https://ethereum.gitbooks.io/frontier-guide/content/contracts_and_transactions_intro.html
			\item http://ecomunsing.com/tutorial-controlling-ethereum-with-python
			\item https://www.npmjs.com/package/ethereumjs-testrpc
 		\end{itemize}
\end{itemize}



\title{December 31st}

Today's goals: finish reading Ethereum documentation, write minimal contract and send it over the private network.


\title{January 1st}

Today's goals: 

\begin{itemize}
	\item finish reading Ethereum documentation.
	\item write minimal contract and send it over the private network.
	\item get a basic REPL workflow for contracts running on EVM 
\end{itemize}


\begin{itemize}
	\item comprehensive step by step of writing a contract: https://auth0.com/blog/an-introduction-to-ethereum-and-smart-contracts-part-2/
	\item an example of a crowdsale: https://www.ethereum.org/crowdsale
\end{itemize}


\title{January 2nd}

Today's goals:

\begin{itemize}
	\item Navigate to a page on node.js website, and do [something] with the contracts
\end{itemize}

Today's accomplishments:

\begin{itemize}
	\item setup private geth network and mined for ether
	\item interacted with private network via a node.js program
\end{itemize}


\title{January 3rd}

Today's goals:

\begin{itemize}
	\item develop a minimum contract and watch how it interacts with the network
	\item develop a node front/backend that pipes the contract around
	\item figure out how the pieces of the system work with each other
\end{itemize}


Today's todos towards goals:

\begin{itemize}
	\item figure out what adress and private keys are 
	\item follow login contract tutorial to get a sense of how things fit together
	\item try sending a contract over the console. learn how contracts compile
	\item make a basic web interface
\end{itemize}

Today's discovered workflow:

\begin{enumerate}
	\item write contract in .sol
	\item load as newline tab stripped string into javascript file
	\item compile in javascript file
\end{enumerate}	

What I learned:

\begin{enumerate}
	\item Contract Defintion: Formal definition in high-level code (e.g. solidity).
	% 
	\item Compiled Contract: The contract converted to byte-code to run on the Ethereum Virtual Machine (EVM), adhering to the specification. Note the function names and input parameters are hashed during compilation. Therefore, for another account to call a function, it must first be given the funtion name and arguments - hence the ABI:
	% 
	\item Application Binary Interface - ABI: A list of the contract's functions and arguments (in JSON1 format). An account wishing to use a smart contract's function uses the ABI to hash the function definition so it can create the EVM bytecode required to call the function. This is then included in the data field, Td, of a transaction and interpreted by the EVM with the code at the target account (the address of the contract).
\end{enumerate}	


sources:

\begin{enumerate}
	\item http://hypernephelist.com/2016/12/13/compile-deploy-ethereum-smart-contract-web3-solc.html
	\item https://medium.com/@mvmurthy/full-stack-hello-world-voting-ethereum-dapp-tutorial-part-1-40d2d0d807c2
	\item solidity voting example: http://solidity.readthedocs.io/en/develop/solidity-by-example.html
\end{enumerate}


\title{January 4th}

sources:

\begin{enumerate}
	\item solidity voting example, very comprehensive: http://solidity.readthedocs.io/en/develop/solidity-by-example.html
\end{enumerate}


\title{January 5th}


What I learned today:

\begin{enumerate}
	\item in order for contracts to be taken by the blockchain, mining has to be running
	\item Events are used in three ways:
		\begin{enumerate}
			\item smart contract return values for the user interface
			\item asynch triggers with data
			\item cheaper form of storage
		\end{enumerate}
	\item I sent a contract to the private blockchain, mutated its value and 
		  saw that it persisted on the chain
\end{enumerate}

workflow:

\begin{enumerate}
	\item start node console: node
	\item run dev ethereum network:  geth --dev --rpc --ipcpath ~/Library/Ethereum/geth.ipc --datadir path/to/data console
	\item run private blockchain with:
	\[geth --nodiscover --datadir data --networkid 123 init genesis.json\]
	and then:
	\[geth --identity "myNode" --datadir path/to/data --networkid 123 --nodiscover --maxpeers 0 --rpc --rpcapi 'web3,eth,admin,personal,shh,debug' --rpcaddr '127.0.0.1' --rpcport 8545 --rpccorsdomain '*' console\]
	\item make sure blockchain is mining with \[miner.start(1)\]
	\item run javascript file: node index.js
\end{enumerate}	


sources:

\begin{enumerate}
	\item solidity voting example, very comprehensive: http://solidity.readthedocs.io/en/develop/solidity-by-example.html
	\item solidity events: https://media.consensys.net/technical-introduction-to-events-and-logs-in-ethereum-a074d65dd61e
	\item how to see transaction details in eth console: https://ethereum.stackexchange.com/questions/6002/transaction-status/6003#6003
	\item https://ethereum.stackexchange.com/questions/30600/struct-value-not-updatinga
	\item view event logs: https://ethereum.stackexchange.com/questions/16313/how-can-i-view-event-logs-for-an-ethereum-contract
\end{enumerate}


\title{January 6th}

What I will accomplish today:

\begin{enumerate}
	\item make front end to display contract, do all CRUD through front end
	\item figure out how to run a minimal node application that interacts with the blockchain
\end{enumerate}

What I accomplished today

\begin{enumerate}
	\item make front end display stuff on node.js
\end{enumerate}

What I learned today:

\begin{enumerate}
	\item Using event listeners to do async call back from blockchain thread
	\item Setting up a project properly with all its dependencies:	
	\begin{enumerate}
		\item 
	\end{enumerate}
\end{enumerate}


sources:

\begin{enumerate}
	\item solidity voting example, very comprehensive: http://solidity.readthedocs.io/en/develop/solidity-by-example.html
	\item solidity events: https://media.consensys.net/technical-introduction-to-events-and-logs-in-ethereum-a074d65dd61e
	\item how to see transaction details in eth console: https://ethereum.stackexchange.com/questions/6002/transaction-status/6003#6003
	\item https://ethereum.stackexchange.com/questions/30600/struct-value-not-updatinga
	\item view event logs: https://ethereum.stackexchange.com/questions/16313/how-can-i-view-event-logs-for-an-ethereum-contract
	\item setup boilerplate express application: http://expressjs.com/en/starter/generator.html
	\item express routes: http://expressjs.com/en/guide/routing.html
\end{enumerate}


\title{January 7th}

What I will accomplish today:

\begin{enumerate}
	\item make front end to display contract, do all CRUD through front end
	\item figure out how to run a minimal node application that interacts with the blockchain
\end{enumerate}

What I accomplished today

\begin{enumerate}
	\item POST from front end template and printed the stuff on the backend
	\item 
\end{enumerate}

What I learned today:

\begin{enumerate}
	\item should set up project the right way with the appropriate dependencies
	\item Using event listeners to do async call back from blockchain thread
	\item connecting POST action from front end HTML to backend express.js app
	\item how to properly install solidity (again): 
		\begin{itemize}
			\item foo
		\begin{itemize}
\end{enumerate}


sources:

\begin{enumerate}
	\item install solidity, this method will actually work when npm install -g solidity fails: 
	\item solidity voting example, very comprehensive: http://solidity.readthedocs.io/en/develop/solidity-by-example.html
	\item solidity events: https://media.consensys.net/technical-introduction-to-events-and-logs-in-ethereum-a074d65dd61e
	\item how to see transaction details in eth console: https://ethereum.stackexchange.com/questions/6002/transaction-status/6003#6003
	\item https://ethereum.stackexchange.com/questions/30600/struct-value-not-updatinga
	\item view event logs: https://ethereum.stackexchange.com/questions/16313/how-can-i-view-event-logs-for-an-ethereum-contract
	\item setup boilerplate express application: http://expressjs.com/en/starter/generator.html
	\item express routes, good clean code: http://expressjs.com/en/guide/routing.html
\end{enumerate}
